\documentclass[12pt]{article}
\usepackage{lingmacros}
\usepackage{tree-dvips}
\usepackage{amsmath,mathpazo}
\begin{document}

\section*{MAT023 - Transformaciones Lineales}

\subsection*{¿Qué es una Transformación Lineal?}

Diremos que una Transformación Lineal será una función que lleva los elementos de 
un subespacio vectorial U hacia otro subespacio V, siempre y cuando cumpla la 
siguiente regla:
\begin{center}
$ T(\alpha u + v) = \alpha T(u) + T(v) $
\end{center}
\\
Esta regla debe cumplirse para cualquier elemento $u \in U$, $v \in U$ y $\alpha \in 
\mathbb{B}$ para que la función sea una Transformación Lineal. Sumado a esto, podemos 
hacer las siguientes observaciones:
\\
\\
1) Sean U y V espacios vectoriales cualquiera. Considere la función 
$T : U \to V$ definida por $T(u) = 0$, para cada $u \in U$
\\
\\
2) Sean U y V espacios vectoriales cualquiera. Considere la función identidad a 
$Id : U \to U$ definida por 
$Id(u) = u$, para cada $e \in U$.
\\
\\
3) El conjunto de todas las Transformaciones Lineales T de U se anota con el símbolo 
$L_\mathbb{K}(U,V)$, esto es:

\begin{center}
$ L_\mathbb{K}(U,V) = \{ T : U \to V \mid$ T es lineal\}
\end{center}
\\
\\
4) Con las operaciones usuales de funciones $L_\mathbb{K}(U,V)$ es un espacio vectorial 
sobre $\mathbb{K}$ 
\\
\\
5) La composición entre Transformaciones Lineales tambíen es una Transformación Lineal
\\
\\
\newpage
\subsection*{Teorema de Existencia y Unicidad}
Sean U y V espacios vectoriales sobre K con Dim $U < \infty$. Suponga que 
$\mathcal{B} = \{u1, u2, ... un\}$
es una base de U y que $v_1, v_2, ... v_n$ son n vectores cualquiera en V, entonces 
existe una única Transformación Lineal $T: U \to V$ tal que:

\begin{center}
	$T(u_i) = v_i, \forall i = 1, 2, ... n.$
\end{center}
\\
Ejemplo: Sea $T : \mathbb{R}^3 \to \mathbb{R}^2$ una función definida a partir de la
relación:

\begin{center}
$T(1,-2,1) = (1,0)$
\\
$T(-2,1,0) = (-1,1)$
\\
$T(1,0,0) = (0,-2)$
\\
\end{center}
\\
¿Es $T$ definida de este modo una Transformación Lineal? En caso afirmativo, hallar
una fórmuña explícita para $T$.
\\
\\
Desarrollo: Asumiremos que es una Transformación Lineal, de esta forma, al generarse 
cualquier inconsistencia, sabremos si realmente es o no es una T.L. Luego, cómo nuestro 
espacio de partida es $\mathbb{R}^3$ definimos las cordenadas de dicho espacio como 
\{x,y,z\}. Para resolver este ejercicio nosotros necesitamos saber cómo la Transformación 
Lineal toma cada una de estas coordenadas y las lleva al espacio de $\mathbb{R}^2$, oséa 
nuestro espacio de llegada.
\\\\
Por el teorema de Existencia y Unicidad, sabemos que solo una función nos dará esas
coordenadas de salida para esas coordenadas de llegada. Entonces en la práctica debemos
buscar cada $T(u_i) = v_i$, siendo cada $u_i$ una coordenada del espacio de salida (para
este caso \{x,y,z\}).
\\\\
Luego, como las Transformaciones Lineales son distribuitivas por definición, podemos
reescribir las transformaciones de la siguiente manera:
\begin{center}
	1) $T(1,-2,1)$ = $T(1,0,0)$ + $T(0,-2,0)$ + $T(0,0,1)$ \\
	2) $T(-2,1,0)$ = $T(-2,0,0)$ + $T(0,1,0)$ + $T(0,0,0)$
\end{center}
Asímismo, podemos sacar las constantes de cada uno de los vectores, para así poder resolver
el sistema. Empezando por la primera transformación entregada:
\begin{center}
	$T(1,-2,1)$ = $T(1,0,0)$ + $T(0,-2,0)$ + $T(0,0,1)$ \\
	$T(1,-2,1)$ = $T(1,0,0)$ + $(-2)*T(0,1,0)$ + $T(0,0,1)$
\end{center}
Como ya sabemos que el resultado de $T(1,0,0) = (0, -2)$ 
\begin{center}
	$T(1,-2,1)$ = $(0,-2)$ + $(-2)T(0,1,0)$ + $T(0,0,1)$ \\
	$(1,0)$ = $(0,-2)$ + $(-2)T(0,1,0)$ + $T(0,0,1)$ \\
	$(1,2)$ = $(-2)T(0,1,0)$ + $T(0,0,1)$
\end{center}
Ahora, trabajemos con la otra transformación:
\begin{center}
	$T(-2,1,0)$ = $T(-2,0,0)$ + $T(0,1,0)$ + $T(0,0,0)$ \\
	$T(-2,1,0)$ = $T(-2,0,0)$ + $T(0,1,0)$ \\
	$T(-2,1,0)$ = $(-2)T(1,0,0)$ + $T(0,1,0)$ \\
	$(-1,1)$ = $(-2)(0,-2)$ + $T(0,1,0)$ \\
	$(-1,1)$ = $(0,4)$ + $T(0,1,0)$ \\
	$(-1,-3)$ = $T(0,1,0)$
\end{center}
Finalmente, como ya tenemos $T(1,0,0)$ y $T(0,1,0)$, solo nos faltaría $T(0,0,1)$
\begin{center}
	$(1,2)$ = $(-2)T(0,1,0)$ + $T(0,0,1)$ \\
	$(1,2)$ = $(-2)(-1,-3)$ + $T(0,0,1)$ \\
	$(1,2)$ = $(2,6)$ + $T(0,0,1)$ \\
	$(-1,-4)$ = $T(0,0,1)$
\end{center}
Entonces habremos hallado la función que define la Transformación Lineal a través
de las propiedades pŕeviamente descritas, resultándonos:
\begin{center}
	$T(1,0,0)$ = (0,-2) \\
	$T(0,1,0)$ = (-1,-3) \\
	$T(0,0,1)$ = (-1,-4)
\end{center}
Descrito de otra forma más explícita, tendremos que para ciertas cordenadas en 
$\mathbb{R}^3$, por ejemplo $\{x,y,z\}$, la Transformación Lineal $T$ se definirá
a través de la siguiente fórmula:
\begin{center}
$T:\mathbb{R}^3\to\mathbb{R}^2$ tal que T$(x,y,z)$ = (-y-z,-2x-3y-4z)
\end{center}
\\
Finalmente, como el sistema es consistente a través del uso de las propiedades
de Transformación Lineal, podemos aseverar que es una T.L.
\newpage
\\
Por supuesto, las Transformaciones Lineales tiene dos características importantes
que permiten identificarlas como tal. Sean $U$ y $V$ espacios vectoriales sobre
$\mathbb{K}$ y $T \in L_\mathbb{K} (U,V)$. Se definen los siguientes conjuntos:
\subsection*{Kernel o Núcleo}
Definición: El núcleo corresponde al conjunto de elementos dentro de $U$ tales que
el resultado de su transformación resulta en el $0V$ (Vector Nulo). Es decir:
\begin{center}
ker $T$ = $\{u \in U: T(u) = 0\} \subseteq U$
\end{center}
Por supuesto, implica que el Vector Nulo está dentro del conjunto $V$.
\subsection*{Imagen}
Definición: La imagen corresponde al conjunto de elementos dentro de $V$ tales que
exista un elemento $u$ dentro del espacio $U$ que al aplicarsele la Transformación
Lineal, resulten en los elementos de $V$. Es decir:
\begin{center}
	Im $T$ = $\{v \in V:\exists u \in U, T(u) = v\} \subseteq V$
\end{center}
\\
\\
Ejemplo: Considere la Transformación Lineal $T : M_2(\mathbb{R}) \to \mathbb{R}^2$ definida:
\begin{center}
T
\begin{pmatrix}
x & y \\
z & w
\end{pmatrix}
= }$(x+y-w,x-y+z)$ 
\end{center}
Calcule el Núcleo e Imagen de T.
\\
\\
Desarrollo: Para hallar el núcleo primero, debemos encontrar los valores de $\{x,y,z,w\}$
tales que su Transformación resulten en el vector $(0,0)$. Entonces realizamos el siguiente
sistema:
\begin{center}
$(x+y-w)$ = $0$ \\
$(x-y+z)$ = $0$ 
\end{center}
\\
Al desarrollar el sistema de ecuaciones como lo haríamos normalmente, obtendremos que:
\begin{center}
$w$ = $-x-y$ \\
$z$ = $-x+y$ \\
\end{center}
\\
Entonces tendremos que el Kernel será cualquier elemento del conjunto $U$ simpre y cuando
$w = -x-y \wedge z = -x+y$, para cualquier $x$ e $y$. Escrito de forma Matemática:

\begin{center}
\therfore ker $T$ = $\{x,y,z,w \in U: z = -x+y, w = -x-y\} \subseteq U$
\end{center}
\\
\\
Luego, como la función de la Transformación ya está definida, la imagen resulta de la
siguiente manera:
\begin{center}
\therfore Im $T$ = $\{v_1,v_2 \in V: v_1 = x+y-w, v_2 = z = x-y-z\} \subseteq V$
\end{center}

Ejemplo: Hallar explícitamente, una transformación lineal $T:\mathbb{R}^3 \to \mathbb{R}^3$
tal que $(1,-2,1)$ $\exists$ $ker$T,$Nu$T = 2 y que además:
\begin{center}
	$Im$T = $\{(x,y,z) \in \mathbb{R}^3; x+2y-z = 0, \wedge y+z = 0\}$
\end{center}
\\
¿Es única esta transformación lineal?


\\
\end{document}
